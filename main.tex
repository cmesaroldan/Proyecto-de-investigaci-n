\documentclass{article}
\usepackage[utf8]{inputenc}



\begin{document}
\begin{center}
\bf{\sc\Huge Aciertos y desaciertos, el nacimiento de una revolución, la computación}\\
\end{center}
\vspace{120pt}
\begin{center}
\bf{\sc\Huge Carlos Andrés Mesa Roldan }\\
\end{center}
\vspace{120pt}
\begin{center}
\bf{\sc\Huge Universidad de Antioquia}\\
\end{center}
\vspace{80pt}
\begin{center}
\bf{\sc\Huge Medellin}
\end{center}
\begin{center}
\bf{\sc\Huge Marzo 26 2020}\\
\end{center}\
\newpage






Computación, un término bastante recurrente hoy por hoy en una infinidad de campos del conocimiento para los cuales se hace reiterativo el uso de computadoras, sin embargo, se ignora en gran medida el origen de este término, su evolución y cómo el significado de éste ha tomado valor gracias a la intervención de diversos personajes quienes a través de la historia y desde su campo -no necesariamente desde el matemático, como muchos así lo afirman- le han aportado significativamente a la manera de abordar problemas por medio de las hoy muy conocidas reglas y operaciones lógicas como también del ejercicio de la abstracción de datos.  
Cuando se habla en un principio acerca de la diversidad de personajes que se relacionan de manera directa e indirecta con el tema de la evolución de la computación, no se miente, pues puede parecer increíble que en la filosofía encontremos respuesta en gran parte al origen de la computación; Y es que en un principio son las cuestiones filosóficas las encargadas de marcar el rumbo de los nuevos intereses de la lógica y de la aplicación de las matemáticas y esto es mucho más evidente en el transcurso del siglo XX. Bertrand Russell, inspirador para coetáneos ligados a las ciencias matemáticas y lógicas del momento fue el principal expositor de diversas paradojas de mucho interés. El tratar los enunciados de manera individual desbordaba sentido y coherencia, más estos carecían de sentido al momento de ser analizados en conjunto. A partir de este momento ya se tenían serias sospechas acerca de la ambigüedad presente en el lenguaje cotidiano. Chaitin (2003) afirma: “A menudo se la llama paradoja de Epiménides, o paradoja del mentiroso. Se dice que Epiménides exclamó: “¡Esta aseveración es falsa!”. ¿Lo es? Si su aseveración es falsa, ha de ser verdadera. Pero, si es verdadera, es falsa”. (p.29). Lo anterior para ser extremista es la ejemplificación del tipo de situaciones y problemáticas a las que durante el siglo XX se estaba en constante exposición y que representaban todo un reto para pensadores del momento. Además, se hace bastante atractiva la manera de abordar ciertas problemáticas que se presentaron en su momento, que, aunque la filosofía y el análisis textual representaban una poderosa herramienta, tenían un límite. 
Seguramente se pensará, y ¿qué paso con las matemáticas y la lógica?: Las matemáticas han sido desde siempre el lenguaje por excelencia en la comprensión del mundo, sin embargo, para muchas personas incluso en el siglo XX era difícil acceder a las mismas, además se les daba uso en campos “formales” de la ciencia y las bases de la lógica -hoy entendida como la manera de abordar problemas en distintos campos, pero en especial en la programación- aun no estaban instauradas. El eco generado por el revolucionario Russell atrajo la atención de pensadores de toda índole, pero en especial de aquellos pertenecientes al área de las matemáticas. David Hilbert fue uno de ellos, el prestigioso matemático propuso la construcción de un conjunto de símbolos y reglas capaces de predecir la veracidad de cualquier sentencia;  Esta propuesta data de la manera en la que los griegos y en particular Euclides con su geometría – se reitera en la influencia de la antigua filosofía- concebían el mundo, una armonía perfecta con la matemática, sin embargo y a pesar de la rigurosidad de la teoría axiomática que esta propuesta conlleva, tenía bastante limitantes que hicieron del sueño de Hilbert algo errado. Y para explicar las limitantes de lo anterior se tiene a Kurt Gödel, quien mediante ejemplos muy precisos explicaba el poco alcance del sistema Hilbert, el cual presentaba ambigüedades incluso en el campo de la aritmética, dejado poco que esperar en los campos de la matemática. Sin embargo, es bastante fructífero la equivocación de Hilbert, primero porque es pionero en la implementación de un lenguaje artificial con la finalidad de solucionar problemas y segundo porque se crea la rama de la matemática denominada metafísica, encargada del estudio del alcance de las matemáticas en el mundo. 
Sin duda los anteriores son grandes legados, aun con las limitantes y errores presentados, es de esta manera como los grandes descubrimientos surgen con base a los errores de predecesores. Los “reproches” presentados de Gödel hacia Hilbert representan un apartado en la historia de la computación popularmente llamado “el teorema de la incompletitud de Gödel”, donde algunos autores señalan que se opaca cualquier relación de la llamada filosofía-matemática con vista hacia el futuro, como bien lo señala Chaitin (2003). “…muchos consideraron que el artículo de Gödel era absolutamente devastador. Toda la filosofía matemática tradicional acababa de quedar reducida a escombros”. (p.31).  Sin embargo, tal acontecimiento pasaría un tanto desapercibido por esos días, pues la atención se centraría en el estallido de una guerra mundial y una devastadora depresión económica. Pero es precisamente en los momentos de mayor crisis donde afloran los más grandes genios, genios como Alan Turing: Turing bastante recordado en el ámbito de la computación – y en muchos otros campos del conocimiento, en especial afines a la matemática-  retoma el trabajo de David Hilbert y establece así las bases de la computación, teniendo como fundamento los diferentes axiomas de campo –que no son nada más ni menos que las diferentes operaciones aritméticas utilizadas de manera cotidiana hoy por hoy – pero con la sutil diferencia de considerar aquel lenguaje que propuso David Hilbert en la solución de problemas, como netamente matemático, y fue así como nació la denominada máquina universal de Turing, con un modelo bastante semejante al funcionamiento de las maquinas computadoras de la actualidad, la máquina de Turing tiene la capacidad de resolver operaciones aritméticas con una increíble velocidad nunca antes vista para aquel momento. Lo anterior sin duda merece un apartado y un distinguido lugar entre los acontecimientos hasta ahora aquí citados. En la actualidad se conoce perfectamente el lenguaje binario y como éste representa la base fundamental en el funcionamiento de una maquina computadora, la máquina de Turing no era ajena a este funcionamiento, fue la primera máquina computadora de la historia, para muchos autores, por el simple hecho de poseer una cinta extremadamente larga  -la cual hace referencia la memoria que hasta el día de hoy se utiliza-  con la capacidad de ser modificada infinidad de veces, conforme el usuario –o manipulador como bien se denominada al operador de la máquina en 1936-  lo requería. Aunque lo anterior representa un increíble avance, hacía falta recorrer aun un frondoso camino, pues si bien en las matemáticas no existe ambigüedad alguna, carecen de algunas cualidades necesarias para determinar si un problema está completamente resuelto, en otras palabras, los números no saben parar y para Turing esto era bastante evidente y perfectamente reflejado en su máquina, incluso por aquellos días se introdujo el término “bucle”-hoy siendo un término bastante utilizado en la programación moderna- haciendo precisamente referencia a aquellos procesos matemáticos carentes de un final determinado. Turing, accidentalmente dedujo uno de los más famosos corolarios en la ciencia de la computación, como bien lo señala Chaitin (2003): “Si no hay forma de determinar de antemano mediante cálculos si un programa va a detenerse o no, tampoco puede haber ningún modo de averiguarlo mediante razonamientos”. (p.33). Con esto se hacen explicitas las limitantes del abuso de los axiomas de campo en la creación de un lenguaje artificial para la solución de problemas. El paso siguiente seria estimar el tiempo para la ejecución y posterior del problema, una labor para nada fácil, así lo determino John Von Neuman, Matemático y físico cuántico estadounidense, quien lograba ser el gran foco de atención en los años posteriores a la guerra. Sin embargo, una gran tarea estaba resuelta y lo que aconteció de allí en adelante involucro en gran medida la física Newtoniana, donde el factor del tiempo para determinar el tiempo en la solución de un problema, jugaría un papel fundamental al igual que la mezcla de las operaciones lógicas con la velocidad y eficiencia de los cálculos matemáticos, donde la toma de decisiones se convierte en el complemento perfecto a lo que hasta el momento se había implementado. Con seguridad a partir del descubrimiento de la máquina de Turing se llevan a cabo importantes contribuciones a la computación y a la manera tan particular de abordar los problemas, las contradicciones y paradojas, más la adaptación de lenguajes artificiales con fundamento en una serie de reglas lógicas, sin embargo, es necesario resaltar algunos detalles puntuales, los cuales fueron olvidados, los cuales la historia no les da el verdadero valor que ameritan, dicho de otro modo, la computación tal y como se le conoce hoy por hoy no habría sido posible sin la intervención de personajes “particulares”, que no necesariamente provenían de las ciencias ligadas a la informática, provenían de campos humanísticos - donde se resaltan la filosofía y las humanidades- como anteriormente se observó, las cuales se encontraron con las ciencias exactas, para generar una bella cooperación que daría como resultado el nacimiento de una necesidad, la cual posteriormente seria saldada por lo que hoy se conoce como la computación.
\newpage






\underline{Bibliografía complementaria}
\begin{enumerate}
 \item Chaitin, G. J. (2003). Ordenadores, paradojas y fundamentos de las matemáticas. Investigación y ciencia, 322, 28-35.
 \item RANDOMNESS AND MATHEMATICAL PROOF. G. J. Chaitin,
en Scientific American, vol. 232, n.o 5, págs. 47-52; 1975.
GÖDEL, ESCHER, BACH: AN ETERNAL GOLDEN BRAID.
D. R. Hofstadter. Basic Books; Nueva York, 1979.


 \end{enumerate} 


\end{document}
